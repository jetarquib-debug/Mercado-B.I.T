from django.db import models

# País
class Pais(models.Model):
    nomb_pais = models.CharField(max_length=100)

    def __str__(self):
        return self.nomb_pais

# Departamentos
class Departamento(models.Model):
    nomb_departamento = models.CharField(max_length=100)
    pais = models.ForeignKey(Pais, on_delete=models.CASCADE)
    def __str__(self):
        return self.nomb_departamento
    
# Provincias
class Provincia(models.Model):
    nomb_provincia = models.CharField(max_length=100)
    departamento = models.ForeignKey(Departamento, on_delete=models.CASCADE)

    def __str__(self):
        return self.nomb_provincia

# Distritos
class Distrito(models.Model):
    nomb_distrito = models.CharField(max_length=100)
    provincia = models.ForeignKey(Provincia, on_delete=models.CASCADE)

    def __str__(self):
        return self.nomb_distrito

# Usuarios
class Usuario(models.Model):
    SEXO_CHOICES = [
        ('M', 'Masculino'),
        ('F', 'Femenino'),
        ('O', 'Otro'),
    ]
    
    nombres = models.CharField(max_length=100, null=True, blank=True)
    apellidos = models.CharField(max_length=100, null=True, blank=True)
    email = models.EmailField(unique=True)
    telefono = models.CharField(max_length=20, null=True, blank=True)
    direccion = models.CharField(max_length=200, null=True, blank=True)
    distrito = models.ForeignKey(Distrito, null=True, on_delete=models.SET_NULL, blank=True)
    provincia = models.ForeignKey(Provincia, null=True, on_delete=models.SET_NULL, blank=True)
    departamento = models.ForeignKey(Departamento, null=True, on_delete=models.SET_NULL, blank=True)
    pais = models.ForeignKey(Pais, null=True, on_delete=models.SET_NULL, blank=True)
    sexo = models.CharField(max_length=1, choices=SEXO_CHOICES, null=True, blank=True)
    fecha_registro = models.DateTimeField(auto_now_add=True)
    contrasena = models.CharField(max_length=255)
    dni_ce = models.IntegerField(null=True, blank=True)

    def __str__(self):
        return f"{self.nombres} {self.apellidos}"

#Imagen de perfil
class ImagenPerfil(models.Model):
    usuario = models.ForeignKey(Usuario, on_delete=models.CASCADE)
    imagen = models.ImageField(upload_to='perfil_imagenes/')
    fecha_subida = models.DateTimeField(auto_now_add=True)
    es_principal = models.BooleanField(default=False)


#Codigo de telefono de pais
class CodigoPais(models.Model):
    pais = models.OneToOneField(Pais, on_delete=models.CASCADE)
    codigo = models.CharField(max_length=10)
    imagen_pais= models.ImageField(upload_to='codigo_pais_imagenes/', null=True, blank=True)

    def __str__(self):
        return f"{self.pais.nomb_pais} (+{self.codigo})"

# Tiendas
class Tienda(models.Model):
    nombre_tienda = models.CharField(max_length=150)
    descripcion = models.CharField(max_length=255, null=True, blank=True)
    email = models.EmailField(unique=True, null=True, blank=True)
    codigo_pais = models.OneToOneField(CodigoPais, null=True, blank=True, on_delete=models.SET_NULL)
    telefono = models.CharField(max_length=20, null=True, blank=True)
    direccion = models.CharField(max_length=200, null=True, blank=True)
    distrito = models.OneToOneField(Distrito, null=True, blank=True, on_delete=models.SET_NULL)
    provincia = models.OneToOneField(Provincia, null=True, blank=True, on_delete=models.SET_NULL)
    pais = models.OneToOneField(Pais, null=True, blank=True, on_delete=models.SET_NULL)
    fecha_registro = models.DateTimeField(auto_now_add=True)
    ruc = models.CharField(max_length=11, null=True, blank=True)
    usuario = models.OneToOneField(Usuario, on_delete=models.CASCADE)
    def __str__(self):
        return self.nombre_tienda

# Marca
class Marca(models.Model):
    nomb_marca = models.CharField(max_length=100, unique=True)
    imagen_marca = models.ImageField(upload_to='marcas_imagenes/', null=True, blank=True)

    def __str__(self):
        return self.nomb_marca

# Categoria
class Categoria(models.Model):
    nomb_ca = models.CharField(max_length=100, unique=True)
    descripcion = models.CharField(max_length=255, null=True, blank=True)
    imagen_categoria = models.CharField(max_length=255, null=True, blank=True)

    def __str__(self):
        return self.nomb_ca

# Productos
class Producto(models.Model):
    ESTADO_CHOICES = [
        ('disponible', 'Disponible'),
        ('agotado', 'Agotado'),
        ('preventa', 'Preventa'),
        ('reacondicionado', 'Reacondicionado'),
        ('descargable', 'Descargable'),
    ]
    
    tienda = models.ForeignKey(Tienda, on_delete=models.CASCADE)
    nomb_prod = models.CharField(max_length=100)
    descripcion = models.CharField(max_length=255, null=True, blank=True)
    precio = models.DecimalField(max_digits=10, decimal_places=2)
    stock = models.IntegerField()
    fecha_creacion = models.DateTimeField(auto_now_add=True)
    peso = models.DecimalField(max_digits=10, decimal_places=2, null=True, blank=True)
    estado = models.CharField(max_length=20, choices=ESTADO_CHOICES)
    marca = models.ForeignKey(Marca, null=True, on_delete=models.SET_NULL)
    categoria=models.ManyToManyField(Categoria)

    def __str__(self):
        return self.nomb_prod

# Métodos de pago
class MetodoPago(models.Model):
    nomb_meto = models.CharField(max_length=50, unique=True)
    imagen_metodo = models.ImageField(upload_to='metodos_pago_imagenes/', null=True, blank=True)

    def __str__(self):
        return self.nomb_meto

#Promociones
class Promocion(models.Model):
    producto = models.ManyToManyField(Producto, on_delete=models.SET_NULL, null=True)
    Categoria = models.ManyToManyField(Categoria, null=True, on_delete=models.SET_NULL)
    descripcion = models.CharField(max_length=255, null=True, blank=True)
    descuento_porcentaje = models.DecimalField(max_digits=5, decimal_places=2)
    fecha_inicio = models.DateTimeField()
    fecha_fin = models.DateTimeField()
    metodo_pago = models.ForeignKey(MetodoPago, null=True, on_delete=models.SET_NULL)
    imagen_promocion = models.ImageField(upload_to='promociones_imagenes/', null=True, blank=True)

# Imagenes de productos
class Imagen_Producto(models.Model):
    producto = models.ForeignKey(Producto, on_delete=models.CASCADE)
    imagen = models.ImageField(upload_to='productos_imagenes/')
    es_principal = models.BooleanField(default=False)
    fecha_subida = models.DateTimeField(auto_now_add=True)

# Carritos
class Carrito(models.Model):
    fecha_creacion = models.DateTimeField(auto_now_add=True)
    usuario = models.ForeignKey(Usuario, null=True, on_delete=models.SET_NULL)

# Detalles de órdenes
class DetalleOrden(models.Model):
    carrito = models.OneToOneField(Carrito, on_delete=models.CASCADE)
    producto = models.ForeignKey(Producto, on_delete=models.CASCADE)
    cantidad = models.IntegerField()
    subtotal = models.DecimalField(max_digits=10, decimal_places=2)

# Órdenes
class Orden(models.Model):
    ESTADO_CHOICES = [
        ('pendiente_envio', 'Pendiente de Envío'),
        ('preparando_envio', 'Preparando Envío'),
        ('en_transito', 'En Tránsito'),
        ('en_reparto', 'En Reparto'),
        ('entregado', 'Entregado'),
        ('cancelado', 'Cancelado'),
    ]
    
    carrito = models.ForeignKey(Carrito, on_delete=models.CASCADE)
    total = models.DecimalField(max_digits=10, decimal_places=2)
    direccion_entrega = models.CharField(max_length=200, null=True, blank=True)
    distrito_entrega = models.OneToOneField(Distrito, null=True, blank=True, on_delete=models.SET_NULL)
    provincia_entrega = models.OneToOneField(Provincia, null=True, blank=True, on_delete=models.SET_NULL)
    pais_entrega = models.OneToOneField(Pais, null=True, blank=True, on_delete=models.SET_NULL)
    fecha_envio = models.DateTimeField(null=True, blank=True)
    fecha_entrega = models.DateTimeField(null=True, blank=True)
    estado = models.CharField(max_length=30, choices=ESTADO_CHOICES)

# Pagos
class Pago(models.Model):
    ESTADO_CHOICES = [
        ('pendiente', 'Pendiente'),
        ('procesando', 'Procesando'),
        ('completado', 'Completado'),
        ('fallido', 'Fallido'),
        ('reembolsado', 'Reembolsado'),
        ('cancelado', 'Cancelado'),
    ]
    
    orden = models.ForeignKey(Orden, on_delete=models.CASCADE)
    metodo = models.ForeignKey(MetodoPago, on_delete=models.SET_NULL)
    estado = models.CharField(max_length=20, choices=ESTADO_CHOICES)
    fecha_pago = models.DateTimeField(null=True, blank=True)

